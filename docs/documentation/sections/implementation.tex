\documentclass[../main.tex]{subfiles}
\graphicspath{{\subfix{../images/}}}

\begin{document}
\section{Implementacja (opcja)}
Program, który rozwijano w trakcie realizacji badania technologii PWA ma cechować się odpowiednim działaniem jako aplikacja webowa i aplikacja natywna w ramach technologii PWA. Ma również stworzyć możliwość rozbudowania o wiele przydatnych funkcji występujących u konkurencji. Celem jest stworzenie oprogramowania w technologii PWA. Chcemy wykorzystać możliwości, które są udostępniane przez tę technologię, czyli celem tego projektu jest aby odczucia z korzystania z aplikacji PWA były takie jak odczucia z korzystania z aplikacji natywnej. Uniemożliwienie skanowania kodów EAN i QR oraz zautomatyzowanie lokalizacji są funkcjonalnościami wyróżniającymi to rozwiązanie na tle konkurencji. Aplikację SNS ma cechować bezpieczeństwo i niwelowanie błędów spowodowanych przez pracowników (złe przeliczenie towaru, błędne zdefiniowanie magazynu). Poprzez dostęp do lokalizacji urządzenia pracownik nie będzie musiał być pewny, w którym magazynie się znajduje - system zrobi to za niego.

    \subfile{implementation-sections/frontend.tex}
    \newpage
    \subfile{implementation-sections/backend.tex}
    \newpage
    \subfile{implementation-sections/aws.tex}      
\end{document}