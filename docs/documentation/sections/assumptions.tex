\documentclass[../main.tex]{subfiles}
\graphicspath{{\subfix{../images/}}}

\begin{document}
\section{Założenia wstępne}
    \subsection{Dobór techonologii}
        Najważniejszą technologią dla projektu jest \textbf{Progressive Web App} (PWA)\cite{pwa}. Celem opracowywanego programu było zapewnienie, aby aplikacja w ramach technologii PWA działała zarówno jako aplikacja webowa, jak i natywna, oferując jednocześnie możliwość rozbudowy o funkcje przydatne użytkownikom oraz te występujące u konkurencji. Kluczowym założeniem projektu było zapewnienie, aby doświadczenie użytkownika podczas korzystania z aplikacji PWA było zbliżone do odczuć płynących z użytkowania aplikacji natywnych.

        W ramach projektu zaplanowano implementację funkcjonalności wyróżniających aplikację na tle konkurencji, takich jak skanowanie kodów EAN i QR oraz wykrywanie lokalizacji urządzenia. Te elementy mają nie tylko zwiększyć wygodę użytkowania, ale także wpłynąć na efektywność procesów magazynowych. Dzięki wykorzystaniu lokalizacji urządzenia system automatycznie identyfikuje magazyn, w którym znajduje się pracownik, eliminując konieczność manualnego wyboru. Funkcjonalność ta minimalizuje błędy ludzkie, takie jak niepoprawne przeliczanie towaru czy błędne definiowanie lokalizacji magazynowej. Ważnym aspektem aplikacji SNS jest także zapewnienie wysokiego poziomu bezpieczeństwa oraz redukcja błędów związanych z obsługą systemu.

        Podczas wyboru frameworka do implementacji warstwy użytkownika (frontendu) kluczowym kryterium było wsparcie dla technologii PWA. Ostatecznie wybrano \textbf{Angular}\cite{angular}. Framework ten, dzięki swojej kompleksowości, wbudowanym narzędziom oraz rozbudowanemu ekosystemowi, umożliwia szybkie i efektywne tworzenie nowoczesnych aplikacji PWA spełniających wysokie wymagania użytkowników.

        Warstwę systemową (backend) zaimplementowano przy użyciu frameworku \textbf{Spring Boot}\cite{springboot}, który ma duże wsparcie dla zabezpieczeń (Spring Security) oraz prosty do zaimplementowania zestaw punktów końcowych (endpointów) jako REST API.

        Jako bazę danych wybrano \textbf{PostgreSQL}\cite{postgresql}. Zespół nie miał doświadczenia z nierelacyjnymi bazami danych, tylko taka alternatywa była dostępna na AWS.

        Całość architektury aplikacji została wdrożona przy pomocy \textbf{Terraform HashiCorp}\cite{terraform} na serwery \textbf{Amazon Web Services} (AWS)\cite{aws}, co niskim kosztem pozwala na przetestowanie różnych rozwiązań bez konieczności samodzielnego utrzymania infrastruktury.
    \subsection{Ograniczenia}
        W fazie wstępnej planowania projektu przyjęto wyobrażonego klienta, który ustalił następujące założenia:
        \subsubsection*{Klient}
            \begin{itemize}
                \item Aplikacja wytwarzana jest dla jednego klienta
                \item Klient posiada wiele magazynów
                \item Klient chce być w stanie sprawdzić stan każdego z magazynów
                \item Klient chce dać swoim pracownikom możliwość dodawania artykułów do stanu magazynu
                \item Klient chce mieć wyłączny dostęp do dodawania i modyfikowania artykułów, magazynów i kategorii
            \end{itemize} 
        \subsubsection*{Aplikacja}
            \begin{itemize}
                \item Aplikacja musi wspierać techonolgię PWA
                \item Aplikacja musi działać na popularnych przeglądarkach
                \item Aplikacja musi być instalowalna na różnych urządzeniach
                \item Aplikacja musi wspierać skanowanie kodów kreskowych i kodów QR
                \item Aplikacja musi wspierać autoryzację użytkowników
            \end{itemize}
        \subsubsection*{Inne}
            \begin{itemize}
                \item Infrastruktura musi być skalowalna
                \item Infrastruktura musi być zabezpieczona
            \end{itemize}

\end{document}