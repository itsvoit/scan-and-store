\documentclass[../main.tex]{subfiles}
\graphicspath{{\subfix{../images/}}}

\begin{document}
\section{Założenia wstępne}
    \subsection{Dobór techonologii}
        Najważniejszą techonologią dla projektu jest \textbf{Progressive Web App} (PWA). 
        
        Aby wybrać framework do implementacji warstwy użytkownika (fronendu) zwrócono szczególną uwagę na dobre wsparcie dla technologii PWA i wybrano framework \textbf{Angular}. Dzięki swojej kompleksowości, wbudowanym narzędziom i ekosystemowi, Angular umożliwia szybkie i efektywne budowanie aplikacji PWA spełniające nowoczesne wymagania użytkowników.

        Warstwę systemową (backend) zaimplementowano przy użyciu frameworku \textbf{Spring Boot}, który ma duże wsparcie dla zabezpieczeń (Spring Security) oraz prosty do zaimplementowania zestaw punktów końcowych (endpointów) jako REST API.

        Jako bazę danych wybrano \textbf{PostgreSQL}. Zespół nie miał doświadczenia z nierelacyjnymi bazami danych, tylko taka alternatywa była dostępna na AWS.

        Całość architektury aplikacji została wdrożona przy pomocy \textbf{Terraform HashiCorp} na serwery \textbf{Amazon Web Services} (AWS), co niskim kosztem pozwala na przetestowanie różnych rozwiązań bez konieczności samodzielnego utrzymania infrastruktury.
    \subsection{Ograniczenia}
        W fazie wstępnej planowania projektu przyjęto wyobrażonego klienta, który ustalił następujące założenia:
        \subsubsection*{Klient}
            \begin{itemize}
                \item Aplikacja wytwarzana jest dla jednego klienta
                \item Klient posiada wiele magazynów
                \item Klient chce być w stanie sprawdzić stan każdego z magazynów
                \item Klient chce dać swoim pracownikom możliwość dodawania artykułów do stanu magazynu
                \item Klient chce mieć wyłączny dostęp do dodawania i modyfikowania artykułów, magazynów i kategorii
            \end{itemize} 
        \subsubsection*{Aplikacja}
            \begin{itemize}
                \item Aplikacja musi wspierać techonolgię PWA
                \item Aplikacja musi działać na popularnych przeglądarkach
                \item Aplikacja musi być instalowalna na różnych urządzeniach
                \item Aplikacja musi wspierać skanowanie kodów kreskowych i kodów QR
                \item Aplikacja musi wspierać autoryzację użytkowników
            \end{itemize}
        \subsubsection*{Inne}
            \begin{itemize}
                \item Infrastruktura musi być skalowalna
                \item Infrastruktura musi być zabezpieczona
            \end{itemize}

\end{document}