\documentclass[../main.tex]{subfiles}
\graphicspath{{\subfix{../images/}}}

\begin{document}
    \section{Słownik pojęć (opcja)}
    \begin{itemize}
        \item API (ang. application programming interface) - zestaw poleceń udostępnianych osobom korzystających z danej aplikacji
        \item CORS (ang. Cross-Origin Resource Sharing) - mechanizm umożliwiający współdzielenie zasobów pomiędzy serwerami znajdującymi się w różnych domenach
        \item CRUD (ang. Create, Read, Update, Delete) - cztery podstawowe funkcje w aplikacjach korzystających z pamięci trwałej, które umożliwiają zarządzanie nią
        \item CSRF (ang. Cross-Site Request Forgery) - atak polegający na wykorzystaniu zaufania użytkownika do serwisu internetowego w celu wykonania nieautoryzowanych działań
        \item Docker - otwarte oprogramowanie służące do konteneryzacji, działające jako platforma do tworzenia, wdrażania i uruchamiania aplikacji rozproszonych.
        Narzędzie to pozwala umieścić program oraz jego zależności w lekkim, przenośnym, wirtualnym kontenerze, który można uruchomić na prawie każdym serwerze z systemem opartym na jądrze Linux
        \item DTO (ang. Data Transfer Object) - w programowaniu obiektowym, obiekt który przechowuje tylko pola publiczne, bez metod, służący do przesyłania danych pomiędzy warstwami aplikacji lub systemami
        \item Endpoint - adres URL, pod którym dostępne są zasoby w API
        \item Framework - biblioteka narzędzi wspomagających implementację danego rozwiązania w danym języku
        \item JWT (ang. JSON Web Token) - otwarty standard przemysłowy definiujący sposób wymiany danych w formie JSON pomiędzy stronami, stosowany m.in. w celu autoryzacji
        \item OAuth2 - standardowy protokół autoryzacji dostępu, skoncentrowany na ułatwieniu użytkownikowi przepływu autoryzacji dla aplikacji
        \item Poziom izolacji serializacji - poziom izolacji transakcji w bazie danych, który zapewnia, że transakcje są wykonywane w sposób sekwencyjny, eliminując możliwość wystąpienia konfliktów między równocześnie wykonywanymi transakcjami
        \item Presigned GET URL - link do zasobu w chmurze, który pozwala na pobranie zasobu bez konieczności autoryzacji, z ograniczeniem czasowym.
        \item Resource server - serwer udostępniający chronione zasoby i zdolny do akceptowania i odpowiadania na żądania dotyczące chronionych zasobów przy użyciu tokenów dostępu
        \item REST (ang. representational state transfer) - zbiór zasad i wytycznych o tym jak budować API aplikacji webowych
        \item REST API - API trzymające się zasad REST
    \end{itemize}
\end{document}