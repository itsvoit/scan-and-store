\documentclass[../main.tex]{subfiles}
\graphicspath{{\subfix{../images/}}}

\begin{document}
    \section{Słownik pojęć (opcja)}
    \begin{itemize}
        \item API - zestaw poleceń udostępnianych osobom korzystających z danej aplikacji
        \item Aplikacja natywna/dedykowana - programy zaprojektowane i stworzone z myślą o konkretnej platformie mobilnej, takiej jak Android lub iOS
        \item Aplikacja webowa - inaczej aplikacja internetowa, program kompuerowy, który pracuje na serwerze i komunikuje się poprzez sieć z hostem użytkownika komputera
        \item CORS - mechanizm umożliwiający współdzielenie zasobów pomiędzy serwerami znajdującymi się w różnych domenach
        \item CRUD - cztery podstawowe funkcje w aplikacjach korzystających z pamięci trwałej, które umożliwiają zarządzanie nią
        \item CSRF - atak polegający na wykorzystaniu zaufania użytkownika do serwisu internetowego w celu wykonania nieautoryzowanych działań
        \item Docker - otwarte oprogramowanie służące do konteneryzacji, działające jako platforma do tworzenia, wdrażania i uruchamiania aplikacji rozproszonych.
        Narzędzie to pozwala umieścić program oraz jego zależności w lekkim, przenośnym, wirtualnym kontenerze, który można uruchomić na prawie każdym serwerze z systemem opartym na jądrze Linux
        \item DTO - w programowaniu obiektowym, obiekt który przechowuje tylko pola publiczne, bez metod, służący do przesyłania danych pomiędzy warstwami aplikacji lub systemami
        \item EAN - w tym dokumencie w wersji 13-cyfrowej (EAN 13), popularny kod kreskowy
        \item E-commerce - rodzaj handlu prowadzonego w internecie
        \item Endpoint - adres URL, pod którym dostępne są zasoby w API
        \item Framework - biblioteka narzędzi wspomagających implementację danego rozwiązania w danym języku
        \item IaC - to proces zarządzania zasobami centrum danych komputerowych i ich udostępniania za pośrednictwem plików definicji nadających się do odczytu maszynowego, a nie za pośrednictwem fizycznej konfiguracji sprzętu lub interaktywnych narzędzi konfiguracyjnych
        \item iOS - mobilny system operacyjny opracowany przez firmę Apple Inc.
        \item JWT - otwarty standard przemysłowy definiujący sposób wymiany danych w formie JSON pomiędzy stronami, stosowany m.in. w celu autoryzacji
        \item OAuth2 - standardowy protokół autoryzacji dostępu, skoncentrowany na ułatwieniu użytkownikowi przepływu autoryzacji dla aplikacji
        \item Poziom izolacji serializacji - poziom izolacji transakcji w bazie danych, który zapewnia, że transakcje są wykonywane w sposób sekwencyjny, eliminując możliwość wystąpienia konfliktów między równocześnie wykonywanymi transakcjami
        \item Progressive Web App - progresywna aplikacja internetowa uruchamiana tak jak zwykła strona internetowa, ale umożliwiająca stworzenie wrażenia działania jak natywna aplikacja mobilna lub aplikacja desktopowa. PWA może zostać zainstalowana, wtedy może posiadać własną ikonę na pulpicie oraz być niezależna od przeglądarki.
        \item QR - alfanumeryczny, dwuwymiarowy, matrycowy, kwadratowy kod graficzny, opracowany przez japońskie przedsiębiorstwo Denso Wave
        \item Presigned GET URL - link do zasobu w chmurze, który pozwala na pobranie zasobu bez konieczności autoryzacji, z ograniczeniem czasowym.
        \item Responsywność - responsywność aplikacji PWA odnosi się do jej zdolności do dostosowywania się do różnych rozmiarów ekranów, rozdzielczości i urządzeń, takich jak smartfony, tablety, laptopy czy komputery stacjonarne. Oznacza to, że aplikacja zapewnia spójne i optymalne doświadczenie użytkownika bez względu na platformę czy urządzenie.
        \item Resource server - serwer udostępniający chronione zasoby i zdolny do akceptowania i odpowiadania na żądania dotyczące chronionych zasobów przy użyciu tokenów dostępu
        \item REST - zbiór zasad i wytycznych o tym jak budować API aplikacji webowych
        \item REST API - API trzymające się zasad REST
        \item S3 -  internetowy nośnik danych firmy Amazon, ma prosty w obsłudze interfejs WWW, który umożliwia dostęp do przechowywanych danych i zarządzanie nimi
        \item Service Worker - skrypt działający w tle przeglądarki, który pełni rolę pośrednika między aplikacją a siecią, a także umożliwia wykonywanie określonych zadań bez aktywnej interakcji użytkownika. Jest jednym z kluczowych elementów PWA, który pozwala na realizację funkcji takich jak tryb offline, buforowanie zasobów oraz wysyłanie powiadomień push
        \item Urządzenia mobilne - Smartfon, tablet
    \end{itemize}
\end{document}