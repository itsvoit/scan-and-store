\documentclass[../main.tex]{subfiles}
\graphicspath{{\subfix{../images/}}}

\begin{document}
\section{Stan wiedzy w obszarze przedsięwzięcia (opcja)}

Na rynku istnieją aplikacje umożliwiajace zarządzanie inwentarzem, na przykład: 
\begin{itemize}
    \item Zoho Inventory - umożliwia zarządzanie zamówieniami, magazynem oraz wysyłką. Oferuje integrację z platformami e-commerce, takimi jak Shopify i Amazon, co ułatwia przygotowanie i wysyłkę odpowiednich produktów z magazynu. Cechuje się automatyzacją procesów magazynowych oraz integracją z zewnętrznymi platformami handlowymi.
    \item Fishbowl Inventory - jest skierowane głównie do średnich i dużych przedsiębiorstw. Oferuje zaawansowaną obsługę magazynów, skanowanie kodów kreskowych oraz zarządzanie lokalizacjami magazynowymi. Intuicyjny interfejs, dedykowany do bardziej złożonych operacji, takich jak zarządzanie wieloma magazynami i lokalizacjami.
    \item Sortly - narzędzie do zarządzania inwentarzem z możliwością skanowania kodów QR i EAN. Przeznaczone głównie dla małych i średnich przedsiębiorstw, oferuje szybki dostęp do edycji produktów oraz statusu magazynowego po zeskanowaniu kodu QR. Prosty w obsłudze interfejs i szybkość dostępu do zmiany stanu magazynowego sprawiają, że jest to idealne rozwiązanie dla małych firm. Nie posiada zaawansowanego wsparcia dla dużych przedsięborstw.
    \item Odoo Inventory - zaawansowane oprogramowanie do zarządzania magazynami z wieloma funkcjonalnościami. Posiada zaawansowane opcje konfiguracji, w tym zarządzanie lokalizacjami magazynowymi z podziałem na alejki, półki czy pomieszczenia. Integruje skanowanie kodów kreskowych na paczkach, co ułatwia śledzenie paczek i ich zawartości. Posiada szeroką gamę funkcji, które są szczególnie przydatne dla większych przedsiębiorstw, integracja z innymi modułami Odoo, a także możliwość personalizacji zgodnie z wymaganiami użytkownika.
\end{itemize}

Wszystkie wyżej wymienione aplikacja mają na celu ułatwienie zarządzania magazynami, przy czym Zoho Inventory integruje się z internetowymi sklepami co ułatwia przygotowanie i wysyłanie odpowiednich produktów z magazynu. Fishbowl Inventory i Odoo Inventory również mają zbliżone funkcjonalności ale te rozwiązania są dedykowane większym firmom. Sortly - oprogramowanie, którego zakres jest najbliższy projektowi SNS. Jako jedyne umożliwia szybki dostęp do edycji za pomocą zeskanowania kodu QR. To oprogramowanie również udostępnia integrację ale raczej w celach kontaktowo-organizacyjnych (integracja z Microsoft Teams i Slackiem). Żadne z dyskutowanych rozwiązań nie jest napisane w technologii PWA, ale każde posida swoją mobilną aplikację dostępną w sklepie Google. Wszystkie powyższe programy umożliwiają dostęp do statystyk i raportów, a także integrują skanowanie kodów EAN w celu dostępu do produktów lub ich lokalizacji bądź sprzedaży (Zaho Inventory). Wszystkie powyższe oprogramowania nie oferują bezpiecznej metody wyboru i definiowania lokalizacji magazynu. Oferują jednynie zarządzanie lokalizacją produktu w magazynie. %skrot sns microsoft teams slack. ponizsze witryny

Inne znane aplikacje PWA:
\begin{itemize}
    \item Twitter
    \item Facebook
    \item Instagram
    \item YouTube
    \item Google Photos
    \item Google Drive
\end{itemize}

Na rynku można znaleźć wiele aplikacji napisanych w technologii PWA, które można zainstalować na przykład na komputerze. Często jednak są tylko "skrótami" do odpowiednich wytryn przy czym wymagają ciągłego dostępu do internetu. Szczególnie serwisy społecznościowe umożliwiają pobranie aplikacji w formie PWA, choć czasmi tylko na komputer, co można traktować jako  "skrót" do witryny. Ogólne odczucia korzystania z obecnych aplikacji PWA są gorsze w porównaniu do aplikacji dedykowanych dla danego systemu. Aplikacja PWA Instagrama wygląda i działa gorzej niż dedykowany odpowiednik na system Android. Technologia PWA nie jest wykorzystywana w pełni, a jej popularność jest dość niska.
\end{document}