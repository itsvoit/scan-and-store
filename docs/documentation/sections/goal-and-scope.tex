\documentclass[../main.tex]{subfiles}
\graphicspath{{\subfix{../images/}}}

\begin{document}
\section{Cel i zakres przedsięwzięcia}
    \subsection{Cel}    
    Celem projektu jest zbadanie dojrzałości technologii Progressive Web Application (PWA) poprzez opracowanie i wdrożenie systemu do zarządzania inwentarzem. System będzie wykorzystywał kody kreskowe oraz kody QR do szybszego dostępu do informacji o zasobach i ewentualnej edycji. W ramach przedsięwzięcia zostanie przeprowadzona wstępna analiza technologii PWA, jej możliwości oraz ograniczeń, a także ocena jej przydatności w praktycznych zastosowaniach, takich jak zarządzanie stanem magazynowym.
        

    \subsection{Zakres}
        W fazie planowania projektu ustalono, iż aplikacja końcowa powinna korzystać z funkcjonalości oferowanych przez techonologię PWA, co oznacza, że będzie dostępna z poziomu przeglądarki internetowej, ale jednocześnie umożliwiająca instalację na urządzeniu jako aplikacja natywna. Aplikacja ma również umożliwiać zarządzanie stanem magazynu w podstawowym zakresie. 

        Z funkcjonalości PWA jakie powinny zostać zaimplementowane wybrano dostęp do kamery urządzenia aby skanować kody kreskowe i kody QR w celu szybszego dostępu do informacji na temat produktu lub stanu magazynu. Dostęp do geolokacji podyktowany jest chęcią udoskonalenia i zmniejszenia na podatność błędów wprowadzania odpowiednich danych do danego magazynu. Kolejnym ważnym punktem technologii PWA jest wykorzystanie Service Workera do obsługi trybu online oraz przyśpieszenia wczystywania zasobów (szczególnie zdjęć). Responsywność aplikacji również jest ważnym zagadnieniem. Technologia PWA daje możliwość korzystania z aplikacji nie tylko poprzez przeglądakę ale również możliwe jest zainstalowanie aplikacji bezpośrednio na smartfonie (Android lub IOS), na komputerze jak i również tablecie. 
    % dodac do akronimow i slownika wyrazenia takie jak: service worker, android, IOS i nie wiem, moze urzadzzenia mobilne.
\end{document}