\documentclass[polish,edition=2024]{zpidoc}

\title{
    \iflanguage{polish}
    {Scan and Store\break System do Zarządzania Inwentarzem z Użyciem Kodów Kreskowych i Kodów QR}
    {Scan and Store\break Inventory Management System with Barcode and QR Code Integration}
}
\acronym{SNS}
\supervisor{dr inż. Rafał Palak}
% \members{Adam Klementowski \orcid{0009-0003-8693-3601},Adam Rudnicki \orcid{0009-0006-7009-813X},Adam Skowron \orcid{0009-0009-0543-0038},Wojciech Skrzypiec \orcid{0009-0002-7523-0276}}
\members{Adam Klementowski,Adam Rudnicki,Adam Skowron,Wojciech Skrzypiec}
\keywords{zarządzanie inwentarzem, PWA}
\projectLogo{images/logo_zpi.png}
\techStackIcons{Angular,Spring,AWS,TypeScript,Java,Docker,HashiCorp-Terraform,Chrome,Apple-Safari,Firefox}

\graphicspath{images} % source for images

\usepackage{subfiles} % Best loaded last in the preamble

\begin{document}

\maketitle

\tableofcontents
\newpage

\subfile{sections/acronims}
\subfile{sections/goal-and-scope}
\subfile{sections/dictionary}
\subfile{sections/state-of-the-art}
\subfile{sections/assumptions}
\subfile{sections/requirements}
\subfile{sections/project-design}
\subfile{sections/implementation}
\subfile{sections/results}


% \begin{figure}[h]
%     \centering
%     \includegraphics[width=0.5\linewidth]{images/ZPI-Day-cmyk.pdf}
%     \caption{To jest obrazek}
%     \label{fig:zpi-day}
% \end{figure}


% \bibliographystyle{plain}
% \bibliography{references}


\end{document}